\documentclass[12pt,a4paper,spanish]{article}
\usepackage{babel}
\usepackage[latin1]{inputenc}
\usepackage[pdftex]{graphicx}
\usepackage{listings}

\begin{document}

\begin{figure}
  \centering
    \includegraphics{/Users/julio/Documents/CI3661/HASKELL/logo.jpg}
     \\ Universidad Sim\'on Bol\'ivar
     \\ Depto. De Computaci\'on y Tecnolog\'ia de la Informaci\'on
     \\ CI3661: Laboratorio de Lenguajes de Programaci\'on 
\end{figure} 

\title{Tarea 2}
\author{Williams Mari\~no\\
        Julio De Abreu 05-38072}

\maketitle
\newpage

\begin{enumerate}
\item 
\newline
a) 
\newline
\lstset{language=Haskell, breaklines=True, basicstyle=\footnotesize}
\begin{lstlisting}{frame=single}
Papel :: Origami a
Valle :: a -> Origami a -> Origami a
Compuesto :: Origami a -> Origami a
\end{lstlisting}
\newline
\newline
b) 
\lstset{language=haskell, breaklines=True, basicstyle=\footnotesize}
\begin{lstlisting}{frame=single}
transformarPapel :: b
transformarValle :: a -> b -> b
transformarCompuesto :: b -> b
\end{lstlisting}
\newline
\newline
c) 
\lstset{language=haskell, breaklines=True, basicstyle=\footnotesize}
\begin{lstlisting}{frame=single}
plegarOrigami transPapel transPico trnasValle transCompuesto = plegar
    where
        plegar Papel = transPapel
        plegar (Pico x y) = transPico x (plegar y)
        plegar (Valle x y) = transValle x (plegar y)
        plegar (Compuesto x y) = transCompuesto (plegar x) (plegar y)
\end{lstlisting}
d)
\lstset{language=haskell, breaklines=True, basicstyle=\footnotesize}
\begin{lstlisting}{frame=single}
sumarOrigami :: (Num a) -> Origami a -> a
sumarOrigami = plegarOrigami transPapel transPico transValle transCompuesto
    where
        transPapel = 0
        transPico = (+)
        transValle = (+)
        transCompuesto = (+)
\end{lstlisting}
\newline
\newline
e) 
\lstset{language=haskell, breaklines=True, basicstyle=\footnotesize}
\begin{lstlisting}{frame=single}
aplanarOrigami :: Origami a -> [a]
aplanarOrigami = plegarOrigami transPapel transPico transValle transCompuesto
    where
        transPapel = []
        transPico = (:)
        transValle = (:)
        transCompuesto = (:)
\end{lstlisting}
\newline
\newline
g) n funciones.
\newline
\newline
h) La funcion cons.
\newline
\newline
\emph{Investigaci\'on}
\newline
\newline
a) Tenemos que evaluar la expresi\'on: subs (id const) subs const. Para eso aplicamos la definici\'on de subs: 
subs x y z = x z (y z). Esto nos da la siguiente expresio\'on: (id const) const (subs  const). Ahora aplicando la definici\'on de id, la expresi\'on queda de la siguiente manera: const const (subs const). Finalmente aplicando la definicion de const, la expresi\'on queda as\'i: const.
\newline
\newline 
b) subs (subs (subs const sub id) const sub) const id 
\newline
\newline
c) id = const () sub
\newline
\newline
d) El c\'alculo SKI es una versi\'on reducida del Lambda C\'alculo. Todas las operaciones del c\'alculo SKI vienen expresadas como \'arboles binarios en sus tres siglas: S,K,I (los cuales se les conoce como combinadores). 
\newline
\newline
\indent La relaci\'on que guarda el C\'alculo SKI con las funciones propuestas es la evaluacion de las operaciones de los combinadores. Esto se puede ver as\'i:
\newline
\indent El combinador I retorna un elemento: Ix = x. Esta es la funci\'on Identidad que estaba propuesta.
\newline
\indent El combinador K cuando se le aplica a un argumento X, \'este devuelve la funci\'on constante Kx, que luego aplicado a cualquier argumento, devuelve x. En otras palabras: Kxy = x. Esta es la funci\'on const que estaba propuesta.
\newline
\indent Finalmente, S es un operador de sustituci\'on. \'Este toma tres argumentos, y retorna el primero aplicado al tercero, el cual es luego aplicado al resultado del segundo argumento aplicado al tercero. En otras palabras: Sxyz = xz(yz). Y esta es la funci\'on subs que fue propuesta.
\end{document}