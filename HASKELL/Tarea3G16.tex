\documentclass[12pt,a4paper,spanish]{article}
\usepackage{babel}
\usepackage[latin1]{inputenc}
\usepackage[pdftex]{graphicx}
\usepackage{listings}

\begin{document}

\begin{figure}
  \centering
    \includegraphics{/Users/julio/Documents/CI3661/HASKELL/logo.jpg}
     \\ Universidad Sim\'on Bol\'ivar
     \\ Depto. De Computaci\'on y Tecnolog\'ia de la Informaci\'on
     \\ CI3661: Laboratorio de Lenguajes de Programaci\'on 
\end{figure} 

\title{Tarea 3}
\author{Williams Mari\~no\\
        Julio De Abreu 05-38072}

\maketitle
\newpage

\begin{enumerate}
\item 
\newline
a) 
\newline
\lstset{language=prolog, breaklines=True, basicstyle=\footnotesize}
\begin{lstlisting}{frame=single}
%% Implantacion del predicado sumar
sumar(estrella,X,X).
sumar(X,estrella,X).
sumar(up(X),Y,up(Z)):-sumar(X,Y,Z).
\end{lstlisting}
\newline
\newline
b) 
\lstset{language=prolog, breaklines=True, basicstyle=\footnotesize}
\begin{lstlisting}{frame=single}
%% Implantacion del predicado restar
restar(estrella,X,X).
restar(up(X), Y, Z):-restar(X,Y,Z).
\end{lstlisting}
\newline
\newline
c) 
\lstset{language=prolog, breaklines=True, basicstyle=\footnotesize}
\begin{lstlisting}{frame=single}
%% Implantacion del predicado producto
producto(_,estrella,estrella).
producto(up(estrella),X,X).
producto(X,up(Y),Z):-sumar(X,Y,Z1),producto(X,Z1,Z).
\end{lstlisting}

\emph{Investigaci\'on}
\newline
\newline
a) P(x) \emph{and}\ Q(y)
\newline
P(x) : 3
\newline
Q(y) : 3
\newline
1) P(a) \emph{and}\ Q(a)
\newline
2) P(a) \emph{and}\ Q(b)
\newline
3) P(a) \emph{and}\ Q(c)
\newline
4) P(b) \emph{and}\ Q(a)
\newline
5) P(b) \emph{and}\ Q(b)
\newline
6) P(b) \emph{and}\ Q(c)
\newline
7) P(c) \emph{and}\ Q(a)
\newline
8) P(c) \emph{and}\ Q(b)
\newline
9) P(c) \emph{and}\ Q(c)
\newline
\newline
Total: 9 posibles formulas.
\newline
\newline
b) Tomando las formulas de la parte d), se logro sacar la siguiente cuenta:
\newline
\newline
18+9+18+18+9+27=99 posibles formulas.
\newline
\newline
c) P(f(a)) \emph{and} Q(g(a,b))
\newline
   P(f(a)) \emph{and} Q(g(a,c))
   \newline
   P(f(a)) \emph{and} Q(g(b,c))
   \newline
   P(f(b)) \emph{and} Q(g(a,b))
   \newline
   P(f(b)) \emph{and} Q(g(a,c))
   \newline
   P(f(b)) \emph{and} Q(g(b,c))
   \newline
   P(f(c)) \emph{and} Q(g(a,b))
   \newline
   P(f(c)) \emph{and} Q(g(a,c))
   \newline
   P(f(c)) \emph{and} Q(g(b,c))
   \newline
   P(f(a)) \emph{and} Q(g(b,a))
   \newline
   P(f(a)) \emph{and} Q(g(c,a))
   \newline
   P(f(a)) \emph{and} Q(g(c,b))
   \newline
   P(f(b)) \emph{and} Q(g(b,a))
   \newline
   P(f(b)) \emph{and} Q(g(c,a))
   \newline
   P(f(b)) \emph{and} Q(g(c,b))
   \newline
   P(f(c)) \emph{and} Q(g(b,a))
   \newline
   P(f(c)) \emph{and} Q(g(c,a))
   \newline
   P(f(c)) \emph{and} Q(g(c,b))
   \newline
   \newline
   P(f(a)) \emph{and} Q(f(a))
   \newline
   P(f(a)) \emph{and} Q(f(b))
   \newline
   P(f(a)) \emph{and} Q(f(c))
   \newline
   P(f(b)) \emph{and} Q(f(a))
   \newline
   P(f(b)) \emph{and} Q(f(b))
   \newline
   P(f(b)) \emph{and} Q(f(c))
   \newline
   P(f(c)) \emph{and} Q(f(a))
   \newline
   P(f(c)) \emph{and} Q(f(b))
   \newline
   P(f(c)) \emph{and} Q(f(c))
   \newline
   \newline
   P(g(a,b)) \emph{and} Q(g(a,b))
   \newline
   P(g(a,b)) \emph{and} Q(g(a,c))
   \newline
   P(g(a,b)) \emph{and} Q(g(b,c))
   \newline
   P(g(a,b)) \emph{and} Q(g(b,a))
   \newline
   P(g(a,b)) \emph{and} Q(g(c,a))
   \newline
   P(g(a,b)) \emph{and} Q(g(c,b))
   \newline
   P(g(a,c)) \emph{and} Q(g(a,b))
   \newline
   P(g(a,c)) \emph{and} Q(g(a,c))
   \newline
   P(g(a,c)) \emph{and} Q(g(b,c))
   \newline
   P(g(a,c)) \emph{and} Q(g(b,a))
   \newline
   P(g(a,c)) \emph{and} Q(g(b,c))
   \newline
   P(g(a,c)) \emph{and} Q(g(c,a))
   \newline
   P(g(b,c)) \emph{and} Q(g(a,b))
   \newline
   P(g(b,c)) \emph{and} Q(g(a,c))
   \newline
   P(g(b,c)) \emph{and} Q(g(b,c))
   \newline
   P(g(b,c)) \emph{and} Q(g(b,a))
   \newline
   P(g(b,c)) \emph{and} Q(g(b,c))
   \newline
   P(g(b,c)) \emph{and} Q(g(c,a))
   \newline
   \newline
   P(g(b,a)) \emph{and} Q(g(a,b))
   \newline
   P(g(b,a)) \emph{and} Q(g(a,c))
   \newline
   P(g(b,a)) \emph{and} Q(g(b,c))
   \newline
   P(g(b,a)) \emph{and} Q(g(b,a))
   \newline
   P(g(b,a)) \emph{and} Q(g(c,a))
   \newline
   P(g(b,a)) \emph{and} Q(g(c,b))
   \newline
   P(g(c,a)) \emph{and} Q(g(a,b))
   \newline
   P(g(c,a)) \emph{and} Q(g(a,c))
   \newline
   P(g(c,a)) \emph{and} Q(g(b,c))
   \newline
   P(g(c,a)) \emph{and} Q(g(b,a))
   \newline
   P(g(c,a)) \emph{and} Q(g(b,c))
   \newline
   P(g(c,a)) \emph{and} Q(g(c,a))
   \newline
   P(g(c,b)) \emph{and} Q(g(a,b))
   \newline
   P(g(c,b)) \emph{and} Q(g(a,c))
   \newline
   P(g(c,b)) \emph{and} Q(g(b,c))
   \newline
   P(g(c,b)) \emph{and} Q(g(b,a))
   \newline
   P(g(c,b)) \emph{and} Q(g(b,c))
   \newline
   P(g(c,b)) \emph{and} Q(g(c,a))
   \newline
   \newline   
   Luego, dado que la conjuncion es simetrica, podemos obtener lo siguiente:
   \newline
   Q(f(a)) \emph{and} P(g(a,b))
   \newline
   Q(f(a)) \emph{and} P(g(a,c))
   \newline
   Q(f(a)) \emph{and} P(g(b,c))
   \newline
   Q(f(b)) \emph{and} P(g(a,b))
   \newline
   Q(f(b)) \emph{and} P(g(a,c))
   \newline
   Q(f(b)) \emph{and} P(g(b,c))
   \newline
   Q(f(c)) \emph{and} P(g(a,b))
   \newline
   Q(f(c)) \emph{and} P(g(a,c))
   \newline
   Q(f(c)) \emph{and} P(g(b,c))
   \newline
   \newline
  Finalmente:
  \newline
  \newline
  P(f(a)) \emph{and} Q(g(a,a)
  \newline
P(f(a)) \emph{and} Q(g(b,b)
\newline
P(f(a)) \emph{and} Q(g(c,c)
\newline
P(f(b)) \emph{and} Q(g(a,a)
\newline
P(f(b)) \emph{and} Q(g(b,b)
\newline
P(f(b)) \emph{and} Q(g(c,c)
\newline
P(f(c)) \emph{and} Q(g(a,a)
\newline
P(f(c)) \emph{and} Q(g(b,b)
\newline
P(f(c)) \emph{and} Q(g(c,c)
\newline
P(g(a,a)) \emph{and} Q(f(a))
\newline
P(g(a,a)) \emph{and} Q(g(a))
\newline
P(g(a,a)) \emph{and} Q(g(a))
\newline
P(g(b,b)) \emph{and} Q(g(b))
\newline
P(g(b,b)) \emph{and} Q(g(b))
\newline
P(g(b,b)) \emph{and} Q(g(b))
\newline
P(g(c,c)) \emph{and} Q(g(c))
\newline
P(g(c,c)) \emph{and} Q(g(c))
\newline
P(g(c,c)) \emph{and} Q(g(c))
\newline
   Q(f(a)) \emph{and} P(g(b,a))
   \newline
   Q(f(a)) \emph{and} P(g(c,a))
   \newline
   Q(f(a)) \emph{and} P(g(c,b))
   \newline
   Q(f(b)) \emph{and} P(g(b,a))
   \newline
   Q(f(b)) \emph{and} P(g(c,a))
   \newline
   Q(f(b)) \emph{and} P(g(c,b))
   \newline
   Q(f(c)) \emph{and} P(g(b,a))
   \newline
   Q(f(c)) \emph{and} P(g(c,a))
   \newline
   Q(f(c)) \emph{and} P(g(c,b))
   \newline
   \newline
d)
\newline
\lstset{language=prolog, breaklines=True, basicstyle=\footnotesize}
\begin{lstlisting}{frame=single}
p(f(X)) :- q(X, Y), r(Y).
q(g(X, Y), Z) :- r(X), r(Z), q(f(Z), a).
q(X, a).
r(f(f(b))).
r(c).
\end{lstlisting}
\newline
\newline
\indent No se puede encontrar un modelo para el programa dado que en la primera f\'ormula,  ya que el int\'rprete al evaluar q(X,Y), dado que Y no esta unificado con ning\'un valor, \'el verifica que exista un predicado emph{q} con dos par\'ametros, como de hecho existe, y por consiguiente va a intentar unificar Y con a. Luego cuando evalue r(a) falla porque r(a) no existe como predicado verdadero.
\newline
\newline
e) r(f(f(b)) y r(c).
\newline
\newline
f) q(X,a).
\newline
\newline
g) H0 = Todo predicado en el conjunto P de predicados no debe tener variables libres, ni antecedentes. En otras palabras, deben ser hechos.
\newline
 H1 = Todo predicado en el conjunto P de predicados puede tener variables libres, pero deben ser hechos.
 \newline
 Hk = Por cada predicado en el consecuente de cada conjunto de predicados, aplicar recursivamente la formula. K = min {k1,k2,kp} donde p es el numero de predicados en el consecuente. 
\newline
\newline
i) El nivel de q es 0. Si se toma el nivel de los not en  p  y en r como 0, entonces ambos esta\'an a nivel 1. Si se toma al 1, entonces ambos estan al nivel 2.
\newline
\newline
j) Para el nivel 0:
\newline
Hk(q(X,Y) :- r(X), p(Y)) = Hk(r(X)) U Hk(p(y))
\newline
\indent Hk(r(X)) = Hk-1(not(q(X,b)))
   \newline
\indent Hk(p(y)) = HK(q(X,Y) U Hk-1(not(q(X,X))) U Hk-1(not(r(Y)))
   \newline
   \newline
\indent Para el nivel 1:
  \newline
Hk(p(X)) U Hk(r(X))
\newline
\indent Hk(p(X)) = Hk(q(X,Y) U Hk-1(not(q(X,X))) U Hk-1(not(r(Y)))
\indent Hk(r(X)) = Hk-1(not(q(X,b)))
\newline  
k) Esto se ve afectado debido a que va hacer la recursion de manera infinita.
\end{document}